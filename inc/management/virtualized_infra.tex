\subsection{Виртуализированная инфраструктура}
''Виртуализированная инфраструктура'' --- специальный обобщающий термин, введённый в этой работе для объединения двух способов организации облачной инфраструктуры:
\begin{itemize}
    \item при помощи контейнерной организации приложений;
    \item с использованием виртуальных машин для организации приложений.
\end{itemize}
Виртуализированной инфраструктурой в данной работе называется такая инфраструктура, которая:
\begin{enumerate}
    \item Управляет вычислительными мощностями.
    \item Предоставляет возможность запускать программные приложения на вычислительных мощностях этой инфраструктуры.
    \item Позволяет выделять каждому из запущенных программных приложений строго определённое количество ресурсов. 
    Количество ресурсов при этом необязательно кратно количеству аппаратных ресурсов в этой инфраструктуре.
    \item Предоставляет возможность изменять количество выделенных ресурсов в случае, если потребности приложения изменились.
    \item Позволяет запускать дополнительные программные приложения на свободных вычислительных мощностях.
\end{enumerate}
