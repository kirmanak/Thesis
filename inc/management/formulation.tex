\subsection{Постановка задачи}
\label{formulation}
В результате обзора предметной области было выявлено, что в настоящее время не существует решений, удовлетворяющих всем предъявляемым требованиям.
В связи с этим возникает необходимость разработки подхода к управлению виртуализированной инфраструктурой, удовлетворяющего таким требованиям.

Подход к управлению виртуализированной инфраструктурой должен решать задачу взаимодействия с платформами облачных вычислений, в то время как компоненты управления ресурсами будут принимать решения о необходимости осуществления такого взаимодействия.
Таким образом, компонент управления ресурсами получает возможность управлять ресурсами систем с различными технологиями виртуализации инфраструктуры без значительных доработок, потому что задача адаптации новой платформы облачных вычислений будет решена разработанным подходом к управлению.

Для принятия решения об осуществлении управляющего взаимодействия, компоненту управления ресурсами необходимо знать какое количество ресурсов сейчас простаивает\cite{fake-32}.
Для этого подход к управлению должен предусмотреть возможность сбора статистики использования вычислительных мощностей каждым из приложений\cite{fake-33}.

Суммируя вышесказанное, можно сказать, что задача управления виртуализированной инфраструктурой включает в себя две основных подзадачи.
\begin{enumerate}
    \item Предоставление доступа к статистике использования вычислительных мощностей каждым из приложений.
    \item Предоставление возможности изменять количество выделенных вычислительных мощностей каждому из приложений.
\end{enumerate}
