\subsection{Постановка задачи}
Виртуализированной инфраструктурой в данной работе называется такая инфраструктура, которая:
\begin{enumerate}
    \item содержит вычислительные мощности;
    \item предоставляет возможность запускать программные приложения на вычислительных мощностях этой инфраструктуры;
    \item позволяет выделять каждому из запущенных программных приложений только необходимое количество ресурсов, при этом необязательно кратное количеству аппаратных машин в этой инфраструктуре;
    \item предоставляет возможность уменьшать количество выделенных ресурсов в случае, если приложение требует меньшего количества, чем ему предоставлено;
    \item позволяет запускать дополнительные программные приложения на свободных вычислительных мощностях.
\end{enumerate}

Чаще всего такая гибкость в выделении ресурсов достигается с помощью запуска нескольких виртуальных машин на одной аппаратной машине. 
Это позволяет гибко настраивать доступные приложению ресурсы с помощью настройки доступных каждой из виртуальных машин ресурсов.

Одной из ключевых особенностей такой инфраструктуры является возможность автомасштабирования\cite{portable-autoscaler-for-managing-multi-cloud-elasticity}.
Во многих популярных реализациях такая возможность существует как часть виртуализированной инфраструктуры.
В то же время, различные пользователи платформ облачных вычислений реализуют свои сервисы автомасштабирования.
Это позволяет учесть особенности решаемых ими задач и специфику их бизнеса.
В настоящее время такие корпоративные сервисы автомасштабирования могут быть интегрированы только с той реализацией виртуализированной инфраструктуры, для которой они были разработаны изначально, так как программные интерфейсы (API) разных платформ облачных вычислений могут существенно отличаться.
В случае, если появляются требования к смене платформы облачных вычислений, сервис автомасштабирования должен быть либо существенно переработан либо создан заново.

Описанная проблема может быть решена при помощи промежуточного звена между виртуализированной инфраструктурой и сервисом автомасштабирования.
Это промежуточное звено решает задачу управления виртуализированной инфраструктурой.
Таким образом сервис автомасштабирования получает возможность управлять любой виртуализированной инфраструктурой без каких-либо доработок, потому что задача адаптации новой платформы облачных вычислений будет решена сервисом управления.

Кроме управления, для автомасштабирования необходимо знать какое количество ресурсов сейчас простаивает.
Для этого сервис управления должен предоставлять возможность сбора статистики использования вычислительных мощностей каждым из приложений.
На основе этой статистики и будет осуществляться автомасштабирование.

Более формально, задача управления виртуализированной инфраструктурой включает в себя две основных подзадачи:
\begin{enumerate}
    \item Предоставление доступа к статистике использования вычислительных мощностей каждым из приложений (виртуальных машин).
    \item Предоставление возможности изменять количество выделенных вычислительных мощностей каждому из приложений (виртуальных машин).
\end{enumerate}

Таким образом, разрабатываемый сервис управления виртуализированной инфраструктурой должен:
\begin{enumerate}
    \item Предоставлять унифицированный единый программный интерфейс (API) не зависящий от управляемой виртуализированной инфраструктуры.
    \item Быть адаптируемым к новым реализациям виртуализированной инфраструктуры с использованием программных адаптеров.
    \item Позволять собирать статистику по используемым каждым из приложений вычислительных ресурсов.
    \item Позволять изменять количество доступных каждому из приложений вычислительных ресурсов.
\end{enumerate}

