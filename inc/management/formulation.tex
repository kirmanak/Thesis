\subsection{Постановка задачи}
\label{formulation}
В результате обзора предметной области было выявлено, что в настоящее время не существует решений, удовлетворяющих всем предъявляемым требованиям.
В связи с этим, в данной работе будет предложено решение на базе введения в систему промежуточного сервиса, называемого сервисом управления виртуализированной инфраструктурой.

''Виртуализированная инфраструктура'' --- специальный обобщающий термин, введённый в этой работе для объединения двух способов организации облачной инфраструктуры:
\begin{itemize}
    \item способы на базе контейнерной организации приложений;
    \item способы с использованием виртуальных машин для организации приложений.
\end{itemize}
Виртуализированной инфраструктурой в данной работе называется такая инфраструктура, которая:
\begin{enumerate}
    \item Управляет вычислительными мощностями.
    \item Предоставляет возможность запускать программные приложения на вычислительных мощностях этой инфраструктуры.
    \item Позволяет выделять каждому из запущенных программных приложений строго определённое количество ресурсов. 
    Количество ресурсов при этом необязательно кратно количеству аппаратных ресурсов в этой инфраструктуре.
    \item Предоставляет возможность изменять количество выделенных ресурсов в случае, если потребности приложения изменились.
    \item Позволяет запускать дополнительные программные приложения на свободных вычислительных мощностях.
\end{enumerate}

Сервис управления виртуализированной инфраструктурой будет решать задачу взаимодействия с виртуализированной инфраструктурой, в то время как сервисы автомасштабирования будут принимать решения о необходимости осуществления взаимодействия.
Таким образом, сервис автомасштабирования получает возможность управлять любой виртуализированной инфраструктурой без каких-либо доработок, потому что задача адаптации новой платформы облачных вычислений будет решена сервисом управления.

Кроме управления, для осуществления автомасштабирования необходимо знать какое количество ресурсов сейчас простаивает\cite{fake-32}.
Для этого сервис управления должен предоставлять возможность сбора статистики использования вычислительных мощностей каждым из приложений\cite{fake-33}.
На основе этой статистики и будет осуществляться автомасштабирование.

Более формально, задача управления виртуализированной инфраструктурой включает в себя две основных подзадачи:
\begin{enumerate}
    \item Предоставление доступа к статистике использования вычислительных мощностей каждым из приложений (виртуальных машин).
    \item Предоставление возможности изменять количество выделенных вычислительных мощностей каждому из приложений (виртуальных машин).
\end{enumerate}
