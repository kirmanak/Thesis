\subsection{Облачные платформы}
Для организации своей вычислительной инфраструктуры компании в настоящее время всё чаще выбирают решения, основанные на так называемых облачных платформах~\cite{fake-16}.
Такие решения имеют ряд достоинств, среди которых для данной работы наиболее важно следующее: загрузка аппаратного обеспечения может быть выше при организации инфраструктуры в облачное решение, чем при традиционной организации~\cite{cloud-computing-concepts}.
Загрузка аппаратного обеспечения показывает какое количество аппаратных ресурсов используется относительно общего их количества~\cite{fake-17}.
Чем выше загрузка, тем меньше аппаратных ресурсов работает вхолостую или простаивает.
Как холостая работа, так и простой ведут к прямым убыткам, в связи с чем более высокая загрузка обеспечивает финансовую выгоду~\cite{fake-19}.

В облачных инфраструктурах повышение загрузки достигается с помощью разделения вычислительных ресурсов одного и того же сервера между несколькими задачами~\cite{fake-20}.
В традиционной инфраструктуре такое разделение не применяется по причинам безопасности, а также из-за возможного захвата всех аппаратных ресурсов одной задачей и последующим простаиванием других~\cite{fake-22}.
Таким образом, в традиционной архитектуре для выполнения того же объёма работы нужно больше аппаратных ресурсов, чем в облачной, как проиллюстрировано на рис.~\ref{load-utilization}.

\begin{figure}[hbtp]
    \centering
    \includegraphics[width=\textwidth]{img/trad-cloud.pdf}
    \caption{Сравнение виртуализированной и традиционной организации инфраструктуры}
    \label{load-utilization}
\end{figure}
