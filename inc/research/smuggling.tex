\subsection{Поддержка мульти-облачных систем в нативных облачных приложениях используя платформы эластичных контейнеров}
Существует решение\cite{smuggling}, позволяющее организовать облачную систему на основе разнородных поставщиков услуг облачных сервисов и разнородных платформ эластичных контейнеров.
В качестве примера организации такой облачной системы и доказательств работоспособности решения в статье приведена система следующей конфигурации:
\begin{itemize}
    \item Поставщики облачной инфраструктуры:
    \begin{itemize}
        \item Google Cloud Engine
        \item Amazon AWS
        \item OpenStack (в датацентре университета исследователя)
    \end{itemize}
    \item Платформы эластичных контейнеров:
    \begin{itemize}
        \item Kubernetes
        \item Docker Swarm
    \end{itemize}
\end{itemize}

Само решение состоит из утилиты командной строки, принимающей на вход \textit{желаемое} и \textit{текущее} состояния всей облачной системы. 
Затем утилита производит необходимые для достижения \textit{желаемого} состояния мероприятия.
Таким образом, утилита решает задачи сервиса управления виртуализированной инфраструктуры:
\begin{enumerate}
    \item предоставление унифицированного программного интерфейса (API);
    \item предоставить возможность изменять количество выделенных каждому из приложений ресурсов.
\end{enumerate}

В работе так же освещена проблема интеграции дополнительных реализаций виртуализированной инфраструктуры.
В частности, для интеграции дополнительной платформы облачных вычислений или для интеграции дополнительного поставщика облачной инфраструктуры необходимо реализовать адаптер API дополнительно интегрируемого модуля к API всей системы.
Таким образом, задача предоставления возможности адаптации новых модулей решена.

В работе не освещён вопрос мониторинга \textit{текущего} состояния системы и, как следствие, задача сбора статистики использования приложениями вычислительных ресурсов, не решена.