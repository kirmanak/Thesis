\subsubsection{Федерация облачных сервисов с использованием прокси-слоя виртуального API в распределённом облачном окружении}
Известен способ объединения одной из реализаций виртуализированной инфраструктуры OpenStack в федерацию с другими реализациями с помощью прокси-слоя виртуального API, который представляет другие облака как псевдо-зону доступности в OpenStack\cite{federated-cloud-services}.
Используя этот способ, можно создать систему из нескольких облачных платформ, объединённую в OpenStack. Это позволяет использовать модули управления, мониторинга и так далее из OpenStack с другими платформами без необходимости их изменения.

Таким образом, эта работа решает все задачи сервиса управления виртуализированной инфраструктурой:
\begin{enumerate}
    \item задача предоставления единого интерфейса решена предоставлением интерфейса OpenStack;
    \item задача предоставления возможности добавлять дополнительные платформы облачных вычислений решена при помощи псевдо-зон доступности в OpenStack, куда можно добавить необходимую облачную платформу;
    \item задача сбора статистики решена с помощью сбора статистики в OpenStack;
    \item задача изменения количества выделенных приложению ресурсов так же решается с помощью модуля управления OpenStack.
\end{enumerate}
Как видно из этого перечисления, все задачи решены, но решены при помощи использования дополнительного облачного окружения, которое будет избыточно в той прикладной задаче, для которой предлагается использовать сервис управления виртуализированной инфраструктуры.
Дополнительная облачная платформа внесёт существенные задержки и сложность в итоговую систему, а так же создаст дополнительную нагрузку на вычислительные мощности.
Иными словами, этот способ рекомендуется использовать только в ситуациях, когда OpenStack уже является частью системы, но добавлять его только ради получения единого унифицированного интерфейса будет избыточно.