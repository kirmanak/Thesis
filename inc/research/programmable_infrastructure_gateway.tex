\subsubsection*{Программируемый инфраструктурный шлюз для обеспечения работы гибридных облачных сервисов}
Программируемый инфраструктурный шлюз, используемый для обеспечения работы гибридных облачных сервисов в сетевом окружении\cite{programmable-infrastructure-gateway} разработан с целью обеспечения организации взаимодействия частей гибридного облачного окружения.

Под гибридным облачным окружением в патенте понимается облачное окружение, где часть вычислительных ресурсов являются внутренними, а управление ими осуществляется самой организацией (или частным лицом). 
При этом другая часть вычислительных ресурсов управляется сторонней организацией (или сторонним частным лицом) и является внешней. 
В качестве примера приводится использование публичного облачного сервиса Amazon~S3\textsuperscript{\texttrademark} для хранения архивных данных компании и внутренней инфраструктуры организации для хранения корпоративных данных по текущим операциям.

Инфраструктурный шлюз, как и сервис управления, должен решать задачу предоставления унифицированного интерфейса к различным реализациям облачных окружений. 
Отсюда следует, что разработка описываемого инфраструктурного шлюза, как и сервиса управления виртуализированной инфраструктурой, предполагает разработку так называемых облачных адаптеров. 
Под облачным адаптером здесь понимается программный модуль, управляющий передачей данных и команд из частного облака (внутренних ресурсов) в  публичную (внешнюю) инфраструктуру.

Авторы программируемого инфраструктурного шлюза не решали задачу мониторинга используемых приложениями ресурсов в реальном времени. 
Сервис управления, в свою очередь, должен решать такую задачу.

Сервис управления виртуализированной инфраструктурой, в отличие от описываемого шлюза, должен работать не только с гибридными облачными сервисами, но и в инфраструктуре, состоящей только из частного облака или только из публичного.
При этом сервис управления не позволяет передавать данные между двумя частями инфраструктуры.