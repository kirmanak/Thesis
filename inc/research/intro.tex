%существуют облачные системы, системы контейнерной кластеризации; надо бы там ресурсами управлять; но систем много — управлять тяжело; посмотрим, что придумали пытливые умы; выявили, что у их решений есть недостатки (не удовлетворяют всем сформулированным нами требованиям); поставили задачу.

Для организации своей вычислительной инфраструктуры компании в настоящее время всё чаще выбирают решения, основанные на так называемых облачных платформах.
Такие решения имеют ряд достоинств, среди которых для данной работы наиболее важно следующее: загрузка аппаратного обеспечения может быть выше при организации инфраструктуры в облачное решение, чем при традиционной организации\cite{cloud-computing-concepts}.
Эта эффективность достигается с помощью разделения вычислительных ресурсов одного и того же сервера между несколькими задачами.
В традиционной инфраструктуре такое разделение не применяется по причинам безопасности, а также из-за возможного захвата всех аппаратных ресурсов одной задачей и последующим простаиванием других.
В облачных решениях захват всех аппаратных ресурсов невозможен вследствие ограниченного выделения ресурсов каждой из задач.
В общем случае, это способы ограничения ресурсов делятся на два типа\cite{containers-and-vm-big-data}:
\begin{itemize}
    \item решения на базе виртуальных машин;
    \item решения на базе контейнеров.
\end{itemize}

В обоих подходах количество ресурсов, доступных задаче, управляются с помощью:
\begin{itemize}
    \item количества таких виртуальных машин или контейнеров;
    Этот способ называется ''масштабирование''
    \item количества ресурсов, доступных одной виртуальной машине или одному контейнеру.
\end{itemize}
Масштабирование в настоящее время применяется чаще по ряду причин, среди которых:
\begin{itemize}
    \item возможность выделения на одно приложение (задачу) несколько серверов за счёт запуска виртуальных машин или контейнеров на каждом из них.
    \item более высокая загрузка оборудования, так как в случае отсутствия нагрузки на приложение, будет простаивать только то количество аппаратных ресурсов, которое выделено одной виртуальной машине или контейнеру приложения, другие ресурсы при этом будут высвобождены.
    В случае масштабирования есть возможность выделить одной виртуальной машине или контейнеру минимально необходимое количество ресурсов для обработки минимальной нагрузки на приложение, в то время как при выделении ресурсов, достаточных для обработки пиковой нагрузки, будет наблюдаться простой в обычном режиме работы. 
\end{itemize}

Таким образом, существует задача масштабирования приложений.
Для решения этой задачи существуют технологии, имеющие общее название ''автомасштабирование''\cite{portable-autoscaler-for-managing-multi-cloud-elasticity}.
При помощи таких технологий осуществляется автоматическое масштабирование в зависимости от текущей нагрузки на приложение или других факторов.

Во многие платформы облачных вычислений сервисы автомасштабирования уже встроены\cite{portable-autoscaler-for-managing-multi-cloud-elasticity}, однако существуют сервисы автомасштабирования, являющиеся внешними по отношению к платформе.
Причины появления таких сервисов могут быть разными:
\begin{itemize}
    \item отсутствие решений автомасштабирования в используемой облачной платформе;
    \item более эффективная реализация автомасштабирования у внешнего решения, чем у встроенного.
    Это может быть обусловлено как спецификой конкретного приложения, на которое платформа не рассчитана, так и недостаточной эффективностью встроенного решения.
\end{itemize}
На практике зачастую внешние сервисы автомасштабирования спроектированы и разработаны под одну конкретную платформу облачных вычислений и не могут взаимодействовать с другими.

В случае, если система состоит из нескольких разных облачных платформ или в системе осуществляется миграция с одной облачной платформы на другую, появляется необходимость в существенной доработке сервиса автомасштабирования, если он был спроектирован лишь под одну платформу.
Далее в этой главе будет представлен обзор научных работ, статей и патентов, посвящённых решению описанной проблемы.
