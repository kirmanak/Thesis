\subsection{Метод оркестровки гибридными облачными сервисами}
Метод оркестровки гибридными облачными сервисами\cite{orchestrating-hybrid-cloud-services} позволяет при помощи одного унифицированного интерфейса взаимодействовать с гибридным облачным окружением, состоящем из нескольких разнородных платформ облачных вычислений.
Таким образом, этот метод, как и сервис управления виртуализированной инфраструктурой, предоставляет единый унифицированный интерфейс, но в случае сервиса управления интерфейс скрывает всегда одну платформу облачных вычислений.
Метод оркестровки предполагает создание интерфейса, скрывающего несколько разнородных реализаций виртуализированной инфраструктуры, работающих одновременно и формирующих таким образом единое гибридное облачное окружение.

С учётом вышесказанного, можно заметить, что и метод оркестровки, и сервис управления подразумевают возможность работы с различными реализациями виртуализированной инфраструктуры.
Сюда включаются сразу две задачи:
\begin{enumerate}
    \item Сбор статистики использования ресурсов.
    \item Изменение количества выделенных приложению ресурсов.
\end{enumerate}
При этом метод оркестровки не ставит перед собой именно эти задачи, он ставит перед собой задачу предоставления ''всеобъемлющего'' API, который, будет решать эти задачи, если их решает каждая из платформ в гибридном облачном окружении.

Задача предоставления возможности адаптации дополнительной реализации виртуализированной инфраструктуры к единому унифицированному API ставится как перед методом оркестровки, так и перед сервисом управления виртуализированной инфраструктурой.
