\subsubsection{Программная архитектура сервиса мониторинга гибридных платформ облачных вычислений}
Программная архитектура сервиса мониторинга гибридных платформ облачных вычислений \cite{hybrid-cloud-computing-monitoring-software-architecture} была спроектирована для решения двух основных задач:
\begin{itemize} 
    \item Мониторинг в режиме реального времени. Это может быть использовано в целях так называемого превентивного или планового обслуживания, то есть для проведения ряда мероприятий, необходимого для избежания неожиданных сбоев.
    \item Постфактум мониторинг состояния системы. С помощью такого мониторинга можно обнаружить проблемы в работе системы или даже проблемы, связанные с безопасностью системы.
\end{itemize} 
Гибридность целевой платформы (платформы под мониторингом) заключается в том, что она может быть заменена на другую без необходимости вносить изменения в сам сервис мониторинга. 
Таким образом, задача предоставления унифицированного интерфейса решается как сервисом мониторинга, так и сервисом управления.

Сервис управления виртуализированной инфраструктурой, как и сервис мониторинга, решает задачу сбора статистики в реальном времени. 
При этом сервис мониторинга решает дополнительную задачу постфактум мониторинга, но такой задачи не стоит перед сервисом управления.

Перед сервисом мониторинга не стоит задачи предоставления интерфейса, который позволяет изменять количество выделенных каждому из приложений ресурсов. 
Сервис управления, в свою очередь, должен решать такую задачу и позволять не только узнать сколько ресурсов используется сейчас, но и высвобождать неиспользуемые ресурсы, а также выделять дополнительные.