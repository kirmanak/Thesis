\begin{center}
\section*{ЗАКЛЮЧЕНИЕ}
\addcontentsline{toc}{section}{ЗАКЛЮЧЕНИЕ}
\end{center}
В выпускной квалификационной работе были получены следующие результаты.
%\begin{enumerate}
%    \item Проведён обзор предметной области, в результате которого были выявлены преимущества и недостатки существующих решений задачи управления виртуализированной инфраструктурой, а также типовые подходы к управлению такой инфраструктурой.
%    \item В результате обзора предметной области установлено, что в настоящее время в публичном доступе не существует сервиса управления виртуализированной инфраструктурой, удовлетворяющего всем требованиям, выдвинутым в данной работе к такому сервису.
%    \item Принято решение о разработке нового подхода к управлению виртуализированной инфраструктурой. Для этого поставлена задача, а затем сформулированы технические требования в соответствии с поставленной задачей.
%    \item Разработан сервис управления виртуализированной инфраструктурой, удовлетворяющий всем требованиям, сформулированным в работе и задании.
%    \item Продемонстрировано существенное упрощение реализации компонента управления ресурсами при применении предложенного в данной работе подхода, а именно при введении в разрабатываемую систему сервиса управления как дополнительного компонента.
%    \item Выведены формулы, при помощи которых рассчитывается разница $T$ в трудозатратах на реализацию компонента управления ресурсами в системе, где есть сервис управления и системе, где такого сервиса нет.
%\end{enumerate}
\begin{enumerate}
    \item Выявлены типовые подходы к управлению виртуализированной инфраструктурой, преимущества и недостатки существующих решений задачи управления такой инфраструктурой, сформулированы функциональные и нефункциональные требования к сервису управления и сделан вывод, что ни один из существующих сервисов не удовлетворяет этим требованиям.

    \item Разработан подход к управлению виртуализированной инфраструктурой, а также сервис, реализующий этот подход. Выявлено, что все сформулированные требования удовлетворены.

    \item Сформулированы выражения для вычисления трудозатрат на реализацию компонентов управления ресурсами в системах с сервисом управления и без него, с помощью которых показаны преимущества разработанного подхода. 

    \item На практике разработан сервис управления виртуализированной инфраструктурой и продемонстрировано существенное упрощение процесса реализации компонентов управления ресурсами за счёт использования разработанного подхода.
\end{enumerate}
