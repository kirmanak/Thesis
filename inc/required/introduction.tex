\begin{center}
\section*{ВВЕДЕНИЕ}
\addcontentsline{toc}{section}{ВВЕДЕНИЕ}
\end{center}
\textbf{Актуальность темы.}
Вычислительные мощности многих компаний мира в настоящее либо организованы в частные облака либо арендованы на публичных облачных платформах. 
Это обусловлено скоростью внедрения такой инфраструктуры, а также её дешевизной в обслуживании и аренде (если оборудование арендовано).
Кроме того, публичные облачные платформы позволяют малым и средним предприятиям арендовать дорогостоящее оборудование, оплачивая лишь те ресурсы, которые были действительно использованы.

Несмотря на перечисленные достоинства, целиком полагаться на одну облачную платформу нельзя, так как она может выйти из строя в результате сбоя.
Когда такое случается, компания может потерять доступ ко всей своей инфраструктуре, в результате теряя деньги.
В связи с этим, сервисы, позволяющие организовать систему из нескольких вычислительных облачных платформ (для резервирования от непредвиденных сбоев или для скорости доступа из разных точек мира), в настоящее время активно исследуются и даже патентуются различными организациями и исследователями.

Кроме этого, у таких компаний может возникнуть необходимость мигрировать всю инфраструктуру с одной облачной платформы на другую.
Такая необходимость может быть вызвана рядом причин, среди которых выделяют:
\begin{itemize}
    \item ожидаемое снижение стоимости облачной инфраструктуры;
    \item возросшие потребности в вычислительных мощностях, которые не могут быть удовлетворены текущими решениями.
\end{itemize}
Для таких случаев активно исследуются и разрабатываются программные модули, применение которых не привязывает компанию к конкретной реализации облачной платформы, а позволяет миграцию инфраструктуры в другое облако без доработок программного обеспечения.

С учётом тенденций, описанных выше, сервис управления виртуализированной инфраструктурой является актуальным и востребованным решением.

%\textbf{Степень теоретической разработанности темы.}

\textbf{Цель и задачи исследования.}
Целью работы является уменьшение расходов, возникающих при смене одной реализации виртуализированной инфраструктуры на другую.
Для достижения поставленной цели необходимо решить ряд \textbf{задач}:
\begin{enumerate}
\item Исследовать существующие реализации платформ виртуализации инфраструктуры.
\item Сформулировать задачу управления виртуализированной инфраструктурой.
\item Разработать сервис управления виртуализированной инфраструктурой.
\end{enumerate}

\textbf{Область исследования.}
Проведённое исследование виртуализированной инфраструктуры полностью соответствует специальности «Программная инженерия», а содержание выпускной квалификационной работы --- техническому заданию.

\textbf{Объектом исследования} являются платформы виртуализации инфраструктуры.

\textbf{Предметом исследования} является сервис управления виртуализированной инфраструктурой с поддержкой нескольких существующих реализаций платформ виртуализации инфраструктуры.

\textbf{Теоретическую основу исследования} составляют научные труды отечественных и зарубежных авторов по компьютерным технологиям.

\textbf{Методологическую основу исследования} составляют метод сравнения, метод обобщения, общенаучный метод абстрагирования, а так же эксперимент.

%\textbf{Информационная база исследования.} 

%\textbf{Научная новизна исследования} заключается в создании 

\textbf{Практическая значимость исследования.} Разработанный сервис управления может быть использован как в новых, так и в существующих облачных системах.
Такой сервис позволяет осуществлять мониторинг и контроль разнородных платформ облачных вычислений при использовании единого унифицированного программного интерфейса (API).
Это позволяет использовать одни и те же системы мониторинга и управления с разными платформами облачных вычислений, а так же позволяет заменять одну виртуализированную инфраструктуру на другую без доработок в существующих программных модулях.

% \textbf{Апробация результатов исследования.}

% \textbf{Объем и структура работы.}
% TODO !!!!