\begin{center}
\section*{ВВЕДЕНИЕ}
\addcontentsline{toc}{section}{ВВЕДЕНИЕ}
\end{center}
\textbf{Актуальность темы.}
Сегодня управление вычислительными мощностями многие компании доверяют сервисам виртуализированной инфраструктуры. Под сервисом виртуализированной инфраструктуры понимается TODO. 
Виртуализированная инфраструктура много где используется, а её реализации значительно различаются между собой. Переиспользование сервисов, взаимодействующих с одной из реализаций, является невозможным после перехода на другую реализацию инфраструктуры.

\textbf{Степень теоретической разработанности темы.}

\textbf{Цель и задачи исследования.}
Целью работы является уменьшение расходов, возникающих при смене одной реализации виртуализированной инфраструктуры на другую.
Для достижения поставленной цели необходимо решить ряд \textbf{задач}:
\begin{enumerate}
\item Сформулировать задачу управления виртуализированной инфраструктурой.
\item Исследовать существующие реализации платформ виртуализации инфраструктуры.
\item Спроектировать сервис управления виртуализированной инфраструктурой.
\item Разработать спроектированный сервис управления виртуализированной инфраструктурой.
\end{enumerate}

\textbf{Область исследования.}
Проведённое исследование виртуализированной инфраструктуры полностью соответствует специальности «Программная инженерия», а содержание выпускной квалификационной работы --- техническому заданию.

\textbf{Объектом исследования} являются платформы виртуализации инфраструктуры.

\textbf{Предметом исследования} является сервис управления виртуализированной инфраструктурой с поддержкой нескольких существующих реализаций платформ виртуализации инфраструктуры.

\textbf{Теоретическую основу исследования} составляют научные труды отечественных и зарубежных авторов по компьютерным технологиям.

\textbf{Методологическую основу исследования} составляют 

\textbf{Информационная база исследования.} 

\textbf{Научная новизна исследования} заключается в том, что в настоящее время не существует сервисов управления различными реализациями виртуализированной инфраструктуры. Разработанный сервис может уменьшить расходы на смену используемой платформы виртуализации инфраструктуры. 

\textbf{Практическая значимость исследования} заключается в том, что разработанный сервис может быть интегрирован в существующие облачные вычислительные системы.  

\textbf{Апробация результатов исследования.}

\textbf{Объем и структура работы.}