\begin{center}
\section*{СПИСОК ИСПОЛЬЗОВАННЫХ ИСТОЧНИКОВ}
\addcontentsline{toc}{section}{СПИСОК ИСПОЛЬЗОВАННЫХ ИСТОЧНИКОВ}
\end{center}
\begingroup
\renewcommand{\section}[2]{}%
\begin{thebibliography}{40}
\bibitem{fake-11} 
Petcu, D., Macariu, G., Panica, S., \& Crăciun, C. (2013). Portable cloud applications --- from theory to practice. Future Generation Computer Systems, 29(6), 1417-1430.
\bibitem{fake-15}
Matthew Portnoy. Virtualization Essentials, 2nd Edition. – New York: Wiley / Sybex, 2016. – 336 c.
\bibitem{fake-14}
Мартынчук И. Г., Жмылёв С. А. Архитектура и организация сервисов автомасштабирования в облачных системах // Альманах научных работ молодых ученых Университета ИТМО – 2017.
\bibitem{fake-13}
Singh J. et al. Data flow management and compliance in cloud computing //IEEE Cloud Computing. – 2015. – Т. 2. – No. 4. – С. 24-32.
\bibitem{fake-34}
Managing competing elastic Grid and Cloud scientific computing applications using OpenNebula / S Bagnasco, D Berzano, S Lusso [и др.] // Journal of Physics: Conference Series / IOP Publishing. Т. 664. 2015. С. 022004.
\bibitem{fake-37}
Kaushal Vishonika, Bala Anju. Autonomic Fault Tolerance Using HAProxy in Cloud Enviorment. Ph.D. thesis: The Thapar Institute of Engineering and Technology. 2011.
%\bibitem{fake-12}
%Мартынчук И. Г., Жмылёв С. А. Сравнительный анализ систем для организации облачных вычислений // Сборник трудов VIII научно-практической конференции молодых ученых «Вычислительные системы и сети (Майоровские чтения)» – 2017.
%\bibitem{fake-18}
%Жмылёв С.А., Киреев В.Ю., Мартынчук И.Г. Исследование систем с нестационарными процессами // Сборник тезисов докладов конгресса молодых ученых. – СПб., 2018.
\bibitem{fake-21}
Mr. Ray J Rafaels. Cloud Computing: From Beginning to End. – CreateSpace Independent Publishing Platform, 2015. – 152 c.
\bibitem{fake-16}
Жмылёв С.А., Мартынчук И.Г., Киреев В.Ю. Оценка длины периода нестационарных процессов в облачных системах // VI научно-практическая конференция с международным участием «Наука настоящего и будущего» для студентов, аспирантов и молодых ученых. 2018. С. 41–43.
\bibitem{cloud-computing-concepts}
Naresh Kumar Sehgal, Pramod Chandra P. Bhatt. Cloud Computing: Concepts and Practices --- Springer, 2018. --- 269 с.
\bibitem{fake-17}
Bogatyrev V., Vinokurova M. Control and safety of operation of duplicated computer systems // International Conference on Distributed Computer and Communication Networks / Springer. 2017. С. 331–342.
\bibitem{fake-19}
Мартынчук И.Г., Жмылёв С.А. Модели и методы композиции нестационарных распределений // Сборник тезисов докладов конгресса молодых ученых. – СПб., 2019.
\bibitem{fake-20}
Мартынчук И.Г. Разработка программного комплекса для автоматического управления ресурсами облачной вычислительной системы // Аннотированный сборник научно-исследовательских выпускных квалификационных работ бакалавров Университета ИТМО. 2017. С. 44–46.
\bibitem{fake-22}
Mao M., Humphrey M. A performance study on the vm startup time in the cloud // Cloud Computing (CLOUD), 2012 IEEE 5th International Conference on. – IEEE, 2012. – С. 423-430.
\bibitem{fake-23}
Chhibber A., Batra S. Security analysis of cloud computing //International Journal of Advanced Research in Engineering and Applied Sciences. – 2013. – Т. 2. – No. 3. – С. 2278-6252.
\bibitem{containers-and-vm-big-data}
Q. Zhang, L. Liu, C. Pu, Q. Dou, L. Wu and W. Zhou, ''A Comparative Study of Containers and Virtual Machines in Big Data Environment,'' // 2018 IEEE 11th International Conference on Cloud Computing (CLOUD), San Francisco, CA, 2018, pp. 178-185, doi: 10.1109/CLOUD.2018.00030.
\bibitem{fake-24}
Rodriguez I., Llana L., Rabanal P. A General Testability Theory: Classes, properties, complexity, and testing reductions //IEEE Transactions on Software Engineering. – 2014. – Т. 40. – No. 9. – С. 862-894.
\bibitem{fake-28}
Медведев Алексей. Облачные технологии: тенденции развития, примеры исполнения // Современные технологии автоматизации. 2013. Т. 2. С. 6–9.
\bibitem{fake-25}
He S. et al. Elastic application container: A lightweight approach for cloud resource provisioning //Advanced information networking and applications (aina), 2012 ieee 26th international conference on. – IEEE, 2012. – С. 15-22.
\bibitem{fake-26}
Optimal cloud resource auto-scaling for web applications / Jiang J., Jie Lu, Guangquan Zhang [и др.] // 2013 13th IEEE/ACM International Symposium on Cluster, Cloud, and Grid Computing. IEEE, 2013. С. 58–65.
\bibitem{portable-autoscaler-for-managing-multi-cloud-elasticity} 
Y. Wadia, R. Gaonkar and J. Namjoshi, ''Portable Autoscaler for Managing Multi-cloud Elasticity,'' // 2013 International Conference on Cloud \& Ubiquitous Computing \& Emerging Technologies, Pune, 2013, с. 48-51.
\bibitem{fake-32}
Barr Jeff, Narin Attila, Varia Jinesh. Building fault-tolerant applications on aws // Amazon Web Services. 2011. С. 1–15.
\bibitem{fake-27}
I. Bermudez, S. Traverso, M. Mellia. Exploring the cloud from passive measurements: The Amazon AWS case // 2013 Proceedings IEEE INFOCOM. IEEE, 2013. С. 230–234.
\bibitem{fake-29}
Analytical methods of nonstationary processes modeling / Sergei Zhmylev, Ilya Martynchuk, Valeriy Kireev [и др.] // CEUR Workshop Proceedings. 2019.
\bibitem{fake-30}
Using machine learning for black-box autoscaling / Muhammad Wajahat, Anshul Gandhi, Alexei Karve [и др.] // 2016 Seventh International Green and Sustainable Computing Conference (IGSC) / IEEE. 2016. С. 1–8.
\bibitem{fake-31}
Comparison of open-source cloud management platforms: OpenStack and OpenNebula / Xiaolong Wen, Genqiang Gu, Qingchun Li [и др.] // 2012 9th International Conference on Fuzzy Systems and Knowledge Discovery / IEEE. 2012. С. 2457–2461.
\bibitem{programmable-infrastructure-gateway} 
Патент US 9755858 B2, 2017.
\bibitem{orchestrating-hybrid-cloud-services} 
Патент WO 2014021849 Al, 2014.
\bibitem{hybrid-cloud-computing-monitoring-software-architecture} 
Aktas, MS. Hybrid cloud computing monitoring software architecture. Concurrency Computat Pract Exper. 2018; 30:e4694. https://doi.org/10.1002/cpe.4694 (дата обращения 21.05.2020)
\bibitem{federated-cloud-services} 
M. M. Shreyas, ''Federated Cloud Services using Virtual API Proxy Layer in a Distributed Cloud Environment,'' // 2017 Ninth International Conference on Advanced Computing (ICoAC), Chennai, 2017, с. 134-141.
\bibitem{accentos} 
Патент RU 2020612651, 2020.
\bibitem{smuggling} 
%Kratzke, Nane, ''Smuggling Multi-cloud Support into Cloud-native Applications using Elastic Container Platforms.'' CLOSER. 2017.
Nane Kratzke. 2017. Smuggling Multi-cloud Support into Cloud-native Applications using Elastic Container Platforms. // In Proceedings of the 7th International Conference on Cloud Computing and Services Science (CLOSER 2017). SCITEPRESS - Science and Technology Publications, Lda, Setubal, PRT, 57–70. DOI:https://doi.org/10.5220/0006230700570070 (дата обращения 21.05.2020) \bibitem{supporting-programmable-autoscaling}
Kovács, J. (2019). Supporting Programmable Autoscaling Rules for Containers and Virtual Machines on Clouds. Journal of Grid Computing, 17(4), 813-829.
\bibitem{fake-33}
Lorido-Botrán Tania, Miguel-Alonso José, Lozano Jose Antonio. Auto-scaling techniques for elastic applications in cloud environments // Department of Computer Architecture and Technology, University of Basque Country, Tech. Rep. EHU-KAT-IK-09. 2012. Т. 12. С. 2012.
%\bibitem{fake-35}
%Private IaaS clouds: a comparative analysis of OpenNebula, CloudStack and OpenStack / Adriano Vogel, Dalvan Griebler, Carlos AF Maron [и др.] // 2016 24th Euromicro International Conference on Parallel, Distributed, and Network-Based Processing (PDP) / IEEE. 2016. С. 672–679.
%\bibitem{fake-36}
%Алиев Т. И. Основы моделирования дискретных систем // СПб: СПбГУ ИТМО. 2009. С. 363.
%\bibitem{fake-38}
%Rafaels Ray J. Cloud Computing: From Beginning to End. CreateSpace Independent Publishing Platform, 2015.
%\bibitem{fake-39}
%Rodriguez Ismael, Llana Luis, Rabanal Pablo. A General Testability Theory: Classes, properties, complexity, and testing reductions // IEEE Transactions on software engineering. 2014. Т. 40, No 9. С. 862–894.
%\bibitem{fake-40}
%Huizinga Dorota, Kolawa Adam. Automated defect prevention: best practices in software management. John Wiley \& Sons, 2007.
%\bibitem{fake-41}
%Faynberg Igor, Lu Hui-Lan, Skuler Dor. Cloud computing: Business trends and technologies. John Wiley \& Sons, 2016.
\end{thebibliography}
\endgroup