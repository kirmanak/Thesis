\section[ТЕСТИРОВАНИЕ РАЗРАБОТАННОГО \\ СЕРВИСА УПРАВЛЕНИЯ]{ТЕСТИРОВАНИЕ РАЗРАБОТАННОГО СЕРВИСА УПРАВЛЕНИЯ}
Разработанный сервис управления был протестирован с двумя реализациями виртуализированной инфраструктуры:
\begin{itemize}
    \item ''OpenNebula'' (KVM);
    \item ''Kubernetes''.
\end{itemize}
Все тестовые платформы были установлены на виртуальных машинах, предоставленных сервисом Microsoft Azure.
Каждой из платформ была выделена отдельная независимая виртуальная машина. Характеристики такой виртуальной машины представлены в табл.~\ref{a2-characteristics}
\begin{table}[hbtp]
    \begin{tabularx}{\textwidth}{X r}
        \textbf{Параметр} & \textbf{Значение} \\
        \hline
        Размер ВМ & ''A2\_v2'' \\
        \hline
        Вычислительные ядра, единиц (vCPUs) & 2 \\
        \hline
        Оперативная память, гибибайт (RAM) & 4 \\
        \hline
        Максимальное количество операций ввода-вывода в секунду (IOPS) & 500 \\
        \hline
        Дисковое хранилище, ГиБ & 20 \\
    \end{tabularx}
    \caption{Характеристики тестового окружения}
    \label{a2-characteristics}
\end{table}

При этом в целях тестирования сервиса на платформе ''Kubernetes'' было развёрнуто приложение ''nginx'' версии 1.14.2, а на платформе ''OpenNebula'' был загружен образ ''Ubuntu Minimal 18.04 KVM'' из MarketPlace ''OpenNebula Public''.
Все операции сбора статистики, а также масштабирования приложений производились с этими инсталляциями.