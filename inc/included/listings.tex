\begin{center}
\section*{ПРИЛОЖЕНИЕ 1. ЛИСТИНГИ РАЗРАБОТАННЫХ ПРОГРАММНЫХ МОДУЛЕЙ}
\end{center}
\addcontentsline{toc}{section}{ПРИЛОЖЕНИЕ 1. ЛИСТИНГИ РАЗРАБОТАННЫХ \\ ПРОГРАММНЫХ МОДУЛЕЙ}
В листинге~\ref{api-controller} представлен исходный код компонента ''Менеджер'', который обрабатывает запросы, получаемые от компонента ''REST API''. 
Компонент ''REST API'' предоставлен фреймворком ''Spring''.
\begin{longlisting}
    \caption{Исходный код компонента ''Менеджер'' (ApiController.kt)}
    \label{api-controller}
    \inputminted{kotlin}{code/ApiController.kt}
\end{longlisting}

Листинг~\ref{client-factory} содержит исходный код фабрики объектов типа ''AppClient'', входящей в состав клиентской библиотеки.
С помощью фабрики можно получить доступ к программному интерфейсу поддерживаемой платформы облачных вычислений, имея необходимые параметры подключения.
\begin{longlisting}
    \caption{Исходный код фабрики объектов ''AppClient'' (AppClientFactory.kt)}
    \label{client-factory}
    \inputminted{kotlin}{code/AppClientFactory.kt}
\end{longlisting}

Интерфейс ''AppClient'' представлен в листинге~\ref{client-interface}.
Это унифицированный интерфейс, с помощью которого осуществляется взаимодействие с конкретной платформой облачных вычислений.
\begin{longlisting}
    \caption{Исходный код интерфейса ''AppClient'' (AppClient.kt)}
    \label{client-interface}
    \inputminted{kotlin}{code/AppClient.kt}
\end{longlisting}


Реализация описанного выше интерфейса ''AppClient'' для одной из интегрируемых платформ под названием ''Kubernetes'' представлена в листинге~\ref{kubernetes-client}.
\begin{longlisting}
    \caption{Исходный код реализации интерфейса ''AppClient'' для ''Kubernetes'' (KubernetesClient.kt)}
    \label{kubernetes-client}
    \inputminted{kotlin}{code/KubernetesClient.kt}
\end{longlisting}

Взаимодействие с ''OpenNebula'' осуществляется при помощи ''OpenNebulaClient'', представленного в листинге~\ref{opennebula-client}.
\begin{longlisting}
    \caption{Исходный код реализации интерфейса ''AppClient'' для ''OpenNebula'' (OpenNebulaClient.kt)}
    \label{opennebula-client}
    \inputminted{kotlin}{code/OpenNebulaClient.kt}
\end{longlisting}

Для получения доступа к статистике использования ресурсов каждым конкретным контейнером или виртуальной машиной используется интерфейс ''AppInstance''.
Его исходный код представлен в листинге~\ref{instance-interface};
\begin{longlisting}
    \caption{Исходный код интерфейса ''AppInstance'' (AppInstance.kt)}
    \label{instance-interface}
    \inputminted{kotlin}{code/AppInstance.kt}
\end{longlisting}

Листинг~\ref{kubernetes-instance} содержит реализацию описанного ранее интерфейса ''AppInstance'' для ''Kubernetes''.
\begin{longlisting}
    \caption{Исходный код реализации интерфейса ''AppInstance'' для ''Kubernetes'' (KubernetesInstance.kt)}
    \label{kubernetes-instance}
    \inputminted{kotlin}{code/KubernetesInstance.kt}
\end{longlisting}

Сбор статистики использования ресурсов виртуальной машиной под управлением ''OpenNebula'' может быть осуществлён с помощью ''OpenNebulaInstance'', представленного в листинге~\ref{opennebula-instance}.
\begin{longlisting}
    \caption{Исходный код реализации интерфейса ''AppInstance'' для ''OpenNebula'' (OpenNebulaInstance.kt)}
    \label{opennebula-instance}
    \inputminted{kotlin}{code/OpenNebulaInstance.kt}
\end{longlisting}
