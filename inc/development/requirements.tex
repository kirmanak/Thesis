\subsection{Требования к реализации сервиса управления}
\label{requirements}
К разрабатываемому сервису управления виртуализированной инфраструктурой предъявлен следующий список функциональных и нефункциональных требований:
\begin{enumerate}
    \item Интегрируемость с Kubernetes и OpenNebula.
    Требование интегрируемости именно с этими реализациями виртуализированной инфраструктуры обусловлено существенной разницей в их организации и, соответственно, программных интерфейсах (API), которые они предоставляют.
    Такая существенная разница позволит продемонстрировать преимущества использования сервиса управления наиболее полно.
    \item Адаптируемость к дополнительным реализациям виртуализированной инфраструктуры с использованием программных адаптеров.
    Это нефункциональное требование объясняется необходимостью поддержки различных платформ облачных вычислений, в том числе не существующих на сегодняшний день, в связи с чем должна быть предусмотрена возможность адаптировать сервис к дополнительным платформам. 
    \item Возможность описать и сохранить параметры подключения к виртуализированной инфраструктуре, включая параметры конкретного приложения.
    Это функциональное требование возникает вследствие необходимости поддержки гибридных облачных окружений, состоящих из нескольких облачных платформ и нескольких приложений.
    \item Возможность запросить список текущих приложений, то есть список параметров подключения к каждому из приложений. 
    \item Возможность обновить параметры подключения для каждого из приложений.
    \item Возможность удалить параметры подключения для каждого из приложений.
    \item Возможность запросить информацию по приложению, включающую в себя:
    \begin{itemize}
        \item текущее количество ВМ (виртуальных машин) или контейнеров, выделенное данному приложению в данной платформе облачных вычислений;
        \item имя каждого из контейнеров или ВМ в платформе облачных вычислений;
        \item текущую загрузку центрального процессора в каждом из контейнеров или ВМ;
        \item текущее количество используемой каждой ВМ или контейнером оперативной памяти (RAM).
    \end{itemize}
    Это требование является частью задачи управления, решаемой сервисом управления, а именно подзадача сбора статистики использования вычислительных ресурсов (мониторинга).
    \item Возможность осуществления масштабирования, то есть возможность изменять текущее количество выделенных контейнеров или ВМ для каждого из приложений.
    Это требование является частью задачи управления, решаемой сервисом управления, а именно подзадача изменения количества выделенных вычислительных мощностей.
    \item Предоставление единого унифицированного программного интерфейса (API), не зависящего от типа управляемых виртуализированных инфраструктур, а так же их количества.
\end{enumerate}