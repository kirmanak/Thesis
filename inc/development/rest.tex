\subsection{Описание REST API веб-сервиса}
В этом разделе содержится перечисление всех вызовов HTTP REST API, которые поддерживает веб-сервис, а так же описание запросов и ответов.

\subsubsection*{Запросы}
Передать для сохранения параметры подключения к приложению под управлением ''Kubernetes'' можно с помощью запроса, описание которого приведено в табл.~\ref{create-kubernetes}.

Передать для сохранения параметры подключения к приложению под управлением ''OpenNebula'' можно с помощью запроса, описание которого приведено в табл.~\ref{create-opennebula}.

Обновить параметры ранее созданного подключения к приложению под управлением ''OpenNebula'' можно с помощью запроса, описание которого приведено в табл.~\ref{update-opennebula}.

Обновить параметры ранее созданного подключения к приложению под управлением ''Kubernetes'' можно с помощью запроса, описание которого приведено в табл.~\ref{update-kubernetes}.

Запросить список приложений, параметры подключения к которым сохранены в СУБД, можно с помощью запроса, описание которого приведено в табл.~\ref{get-apps}.

Удалить ранее сохранённые параметры подключения к приложению вне зависимости от платформы, под управлением которой оно находится, можно с помощью запроса, описание которого приведено в табл.~\ref{delete-app}.

Запросить список ВМ или контейнеров относящихся к приложению, параметры подключения к которому ранее были сохранены, вне зависимости от платформы, под управлением которой оно находится, а так же статистическую информацию об использованных ими ресурсах, можно с помощью запроса, описание которого приведено в табл.~\ref{get-app}.

Осуществить масштабирование приложения, параметры подключения к которому ранее были сохранены, вне зависимости от платформы, под управлением которой оно находится, можно с помощью запроса, описание которого приведено в табл.~\ref{scale-app}.

\subsubsection*{Ответы}
Ответ на любой запрос может содержать либо описание ошибки в формате, который является общим для всех запросов, либо содержать тело ответа в формате, зависящем от запроса, либо содержать только HTTP-код ответа без тела сообщения.

Описание общего формата ошибки приведено в табл.~\ref{general-error}.

Ответ на запрос сохранения параметров вне зависимости от платформы, параметры подключения к которой были сохранены, а так же ответ на запрос удаления приложения описан в табл.~\ref{create-delete-response}.

Ответ на запрос списка сохранённых приложений, то есть списка известных параметров подключения к приложениям, описан в табл.~\ref{get-apps-response}.

Ответы на запрос состояния приложения и на запрос масштабирования приложения не зависят от платформы, под управлением которой находится само приложение. Формат ответа описан в табл.~\ref{get-app-response}.

Ответ на запрос обновления конфигурации так же не зависит от платформы облачных вычислений, к которой относится конфигурация.
Формат ответа описан в табл.~\ref{update-response}.
