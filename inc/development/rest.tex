\subsection{Описание REST API веб-сервиса}
В этом разделе содержится перечисление всех вызовов HTTP REST API, которые поддерживает веб-сервис.

Передать для сохранения параметры подключения к приложению под управлением ''Kubernetes'' можно с помощью запроса, описание которого приведено в табл.~\ref{create-kubernetes}.

Передать для сохранения параметры подключения к приложению под управлением ''OpenNebula'' можно с помощью запроса, описание которого приведено в табл.~\ref{create-opennebula}.

Обновить параметры ранее созданного подключения к приложению под управлением ''OpenNebula'' можно с помощью запроса, описание которого приведено в табл.~\ref{update-opennebula}.

Обновить параметры ранее созданного подключения к приложению под управлением ''Kubernetes'' можно с помощью запроса, описание которого приведено в табл.~\ref{update-opennebula}.

Запросить список приложений, параметры подключения к которым сохранены в СУБД, можно с помощью запроса, описание которого приведено в табл.~\ref{get-apps}.

Удалить ранее сохранённые параметры подключения к приложению вне зависимости от платформы, под управлением которой оно находится, можно с помощью запроса, описание которого приведено в табл.~\ref{delete-app}.

Запросить список ВМ или контейнеров относящихся к приложению, параметры подключения к которому ранее были сохранены, вне зависимости от платформы, под управлением которой оно находится, а так же статистическую информацию об использованных ими ресурсах, можно с помощью запроса, описание которого приведено в табл.~\ref{get-app}.

Осуществить масштабирование приложения, параметры подключения к которому ранее были сохранены, вне зависимости от платформы, под управлением которой оно находится, можно с помощью запроса, описание которого приведено в табл.~\ref{scale-app}.

\begin{table}[hbtp]
    \caption{Описание запроса сохранения параметров подключения к ''Kubernetes''-приложению}
    \label{create-kubernetes}
    \begin{tabularx}{\textwidth}{l X}
        Метод & POST \\
        \hline
        Адрес & /api/v1/kubernetes/\{namespace\}/\{deployment\}/ \\
        \hline
        Параметры запроса & { \begin{tabularx}{\linewidth}{l c X}
        \textbf{Имя} & \textbf{Тип} & \textbf{Описание} \\
        \hline
        namespace & String & Namespace, к которому относится приложение. \\
        \hline
        deployment & String & Deployment, к которому относится приложение. \\
        \end{tabularx} } \\
        \hline
        Тело запроса & Конфигурация подключения к ''Kubernetes'' в формате YAML.
    \end{tabularx}
\end{table}

\begin{table}[hbtp]
    \caption{Описание запроса обновления сохранённых параметров подключения к ''Kubernetes''-приложению}
    \label{update-kubernetes}
    \begin{tabularx}{\textwidth}{l X}
        Метод & PUT \\
        \hline
        Адрес & /api/v1/kubernetes/\{namespace\}/\{deployment\}/\{id\}/ \\
        \hline
        Параметры запроса & { \begin{tabularx}{\linewidth}{l c X}
        \textbf{Имя} & \textbf{Тип} & \textbf{Описание} \\
        \hline
        namespace & String & Namespace, к которому относится приложение. \\
        \hline
        deployment & String & Deployment, к которому относится приложение. \\
        \hline
        id & int64 & Идентификатор существующей конфигурации, которая должна быть обновлена. \\
        \end{tabularx} } \\
        \hline
        Тело запроса & Конфигурация подключения к ''Kubernetes'' в формате YAML.
    \end{tabularx}
\end{table}

\begin{table}[hbtp]
    \caption{Описание запроса списка сохранённых приложений}
    \label{get-apps}
    \begin{tabularx}{\textwidth}{l X}
        Метод & GET \\
        \hline
        Адрес & /api/v1/app \\
    \end{tabularx}
\end{table}

\begin{table}[hbtp]
    \caption{Описание запроса удаления сохранённого приложения}
    \label{delete-app}
    \begin{tabularx}{\textwidth}{l X}
        Метод & DELETE \\
        \hline
        Адрес & /api/v1/app/\{id\} \\
        \hline
        Параметры запроса & { \begin{tabularx}{\linewidth}{l c X}
        \textbf{Имя} & \textbf{Тип} & \textbf{Описание} \\
        \hline
        id & int64 & Идентификатор удаляемого приложения. \\
        \end{tabularx} } \\
    \end{tabularx}
\end{table}

\begin{table}[hbtp]
    \caption{Описание запроса состояния сохранённого приложения}
    \label{get-app}
    \begin{tabularx}{\textwidth}{l X}
        Метод & GET \\
        \hline
        Адрес & /api/v1/app/\{id\} \\
        \hline
        Параметры запроса & { \begin{tabularx}{\linewidth}{l c X}
        \textbf{Имя} & \textbf{Тип} & \textbf{Описание} \\
        \hline
        id & int64 & Идентификатор приложения, состояние которого должно быть возвращено. \\
        \end{tabularx} } \\
    \end{tabularx}
\end{table}

\begin{table}[hbtp]
    \caption{Описание запроса масштабирования приложения}
    \label{scale-app}
    \begin{tabularx}{\textwidth}{l X}
        Метод & PATCH \\
        \hline
        Адрес & /api/v1/app/\{id\} \\
        \hline
        Тело запроса & Тело запроса передаётся в формате JSON. { \begin{tabularx}{\linewidth}{l c X}
        \textbf{Имя} & \textbf{Тип} & \textbf{Описание} \\
        \hline
        incrementBy & int64 & Количество ВМ или контейнеров, на которое должно быть увеличены доступные приложению ресурсы. В случае, если число отрицательное, такое количество ВМ или контейнеров приложения будет отключено. \\
        \end{tabularx} } \\
    \end{tabularx}
\end{table}

\begin{table}[hbtp]
    \caption{Описание запроса сохранения параметров подключения к ''OpenNebula''-приложению}
    \label{create-opennebula}
    \begin{tabularx}{\textwidth}{l X}
        Метод & POST \\
        \hline
        Адрес & /api/v1/opennebula/ \\
        \hline
        Тело запроса & Тело запроса передаётся в формате JSON. { \begin{tabularx}{\linewidth}{l c X}
        \textbf{Имя} & \textbf{Тип} & \textbf{Описание} \\
        \hline
        address & String & URL, по которому находится XML RPC API OpenNebula, к которой осуществляется подключение. \\
        \hline
        login & String & Логин пользователя, с которым осуществляется подключение. \\
        \hline
        password & String & Пароль пользователя, с которым осуществляется подключение. \\
        \hline
        password & String & Пароль пользователя, с которым осуществляется подключение. \\
        \hline
        role & int64 & Идентификатор роли, к которой относятся ВМ управляемого приложения. \\
        \hline
        template & int64 & Идентификатор шаблона, по которому создаются новые ВМ управляемого приложения. \\
        \hline
        vmgroup & int64 & Идентификатор группы ВМ, в котороую входят ВМ управляемого приложения. \\
        \end{tabularx} } \\
    \end{tabularx}
\end{table}

\begin{table}[hbtp]
    \caption{Описание запроса обновления параметров подключения к ''OpenNebula''-приложению}
    \label{update-opennebula}
    \begin{tabularx}{\textwidth}{l X}
        Метод & PUT  \\
        \hline
        Адрес & /api/v1/opennebula/\{id\}/ \\
        \hline
        Тело запроса & Тело запроса передаётся в формате JSON. { \begin{tabularx}{\linewidth}{l c X}
        \textbf{Имя} & \textbf{Тип} & \textbf{Описание} \\
        \hline
        address & String & URL, по которому находится XML RPC API OpenNebula, к которой осуществляется подключение. \\
        \hline
        login & String & Логин пользователя, с которым осуществляется подключение. \\
        \hline
        password & String & Пароль пользователя, с которым осуществляется подключение. \\
        \hline
        password & String & Пароль пользователя, с которым осуществляется подключение. \\
        \hline
        role & int64 & Идентификатор роли, к которой относятся ВМ управляемого приложения. \\
        \hline
        template & int64 & Идентификатор шаблона, по которому создаются новые ВМ управляемого приложения. \\
        \hline
        vmgroup & int64 & Идентификатор группы ВМ, в котороую входят ВМ управляемого приложения. \\
        \end{tabularx} } \\
        Параметры запроса & { \begin{tabularx}{\linewidth}{l c X}
        \textbf{Имя} & \textbf{Тип} & \textbf{Описание} \\
        \hline
        id & int64 & Идентификатор конфигурации OpenNebula приложения, которая должна быть обновлена. \\
        \end{tabularx} } \\
    \end{tabularx}
\end{table}

\FloatBarrier