\subsection{Описание REST API веб-сервиса}
В этом разделе содержится перечисление всех вызовов HTTP REST API, которые поддерживает веб-сервис.

\begin{table}[h!]
    \caption{Описание запроса POST /api/v1/kubernetes/\{namespace\}/\{deployment\}/}
    \begin{tabularx}{\textwidth}{l X}
        Метод & POST \\
        \hline
        Адрес & /api/v1/kubernetes/\{namespace\}/\{deployment\}/ \\
        \hline
        Параметры запроса & { \begin{tabularx}{\linewidth}{l c X}
        \textbf{Имя} & \textbf{Тип} & \textbf{Описание} \\
        \hline
        namespace & String & Namespace, к которому относится приложение. \\
        \hline
        deployment & String & Deployment, к которому относится приложение. \\
        \end{tabularx} } \\
        \hline
        Тело запроса & Конфигурация подключения к ''Kubernetes'' в формате YAML.
        Эта конфигурация ещё называется ''kube config''. \\
    \end{tabularx}
\end{table}

\begin{table}[h!]
    \caption{Описание запроса POST /api/v1/opennebula/}
    \begin{tabularx}{\textwidth}{l X}
        Метод & POST \\
        \hline
        Адрес & /api/v1/opennebula/ \\
        \hline
        Тело запроса & Тело запроса передаётся в формате JSON. { \begin{tabularx}{\linewidth}{l c X}
        \textbf{Имя} & \textbf{Тип} & \textbf{Описание} \\
        \hline
        address & String & URL, по которому находится XML RPC API OpenNebula, к которой осуществляется подключение. \\
        \hline
        login & String & Логин пользователя, с которым осуществляется подключение. \\
        \hline
        password & String & Пароль пользователя, с которым осуществляется подключение. \\
        \hline
        password & String & Пароль пользователя, с которым осуществляется подключение. \\
        \hline
        role & int64 & Идентификатор роли, к которой относятся ВМ управляемого приложения. \\
        \hline
        template & int64 & Идентификатор шаблона, по которому создаются новые ВМ управляемого приложения. \\
        \hline
        vmgroup & int64 & Идентификатор группы ВМ, в котороую входят ВМ управляемого приложения. \\
        \end{tabularx} } \\
    \end{tabularx}
\end{table}

\begin{table}[h!]
    \caption{Описание запроса PUT /api/v1/opennebula/\{id\}/}
    \begin{tabularx}{\textwidth}{l X}
        Метод & PUT  \\
        \hline
        Адрес & /api/v1/opennebula/\{id\}/ \\
        \hline
        Тело запроса & Тело запроса передаётся в формате JSON. { \begin{tabularx}{\linewidth}{l c X}
        \textbf{Имя} & \textbf{Тип} & \textbf{Описание} \\
        \hline
        address & String & URL, по которому находится XML RPC API OpenNebula, к которой осуществляется подключение. \\
        \hline
        login & String & Логин пользователя, с которым осуществляется подключение. \\
        \hline
        password & String & Пароль пользователя, с которым осуществляется подключение. \\
        \hline
        password & String & Пароль пользователя, с которым осуществляется подключение. \\
        \hline
        role & int64 & Идентификатор роли, к которой относятся ВМ управляемого приложения. \\
        \hline
        template & int64 & Идентификатор шаблона, по которому создаются новые ВМ управляемого приложения. \\
        \hline
        vmgroup & int64 & Идентификатор группы ВМ, в котороую входят ВМ управляемого приложения. \\
        \end{tabularx} } \\
        Параметры запроса & { \begin{tabularx}{\linewidth}{l c X}
        \textbf{Имя} & \textbf{Тип} & \textbf{Описание} \\
        \hline
        id & int64 & Идентификатор конфигурации OpenNebula приложения, которая должна быть обновлена. \\
        \end{tabularx} } \\
    \end{tabularx}
\end{table}
