\subsection{Реализованные требования}
Таблица~\ref{implemented-req} содержит краткий перечень реализованных требований к сервису управления виртуализированной инфраструктурой.
    
\begin{xltabular}{\textwidth}{p{0.25\textwidth} X}
    \caption{Перечень реализованных требований к сервису} \label{implemented-req} \\ 
    \multicolumn{1}{c}{\textbf{Требование}} & \multicolumn{1}{c}{\textbf{Описание реализации}} \\ \hline 
    \endfirsthead
    
    \multicolumn{2}{c}%
    {\tablename\ \thetable{} -- продолжение с предыдущей страницы} \\
    \multicolumn{1}{c}{\textbf{Требование}} & \multicolumn{1}{c}{\textbf{Описание реализации}} \\ \hline 
    \endhead

    \hline \multicolumn{2}{r}{{Продолжение на следующей странице}} \\ 
    \endfoot

    \hline
    \endlastfoot
    
    Интегрируемость с Kubernetes и OpenNebula. & Реализация интеграции с обеими платформами включена в модуль ''клиентская библиотека''. \\
    \hline
    Адаптируемость к дополнительным платформам. & Паттерн программирования ''Фабрика'', применённый для реализации клиентской библиотеки, позволяет неограниченно расширять список поддерживаемых платформ. \\
    \hline
    Возможность описания и сохранения параметров подключения. & При помощи запросов к REST API веб-сервиса, описанных в табл.~\ref{create-kubernetes} и табл.~\ref{create-opennebula}, передаются параметры подключения для хранения в БД в формате, описанном в табл.~\ref{db-scheme-kub} и табл.~\ref{db-scheme-one} соответственно. \\
    \hline
    Возможность запроса списка текущих приложений. & Список известных параметров подключения предоставляется по REST API с помощью запроса, описанного в табл.~\ref{get-apps}, ответ на который приведён в табл.~\ref{get-apps-response}. \\
    \hline
    Возможность обновления параметров подключения. & Сохранённые ранее параметры обновляются с помощью запросов к REST API, описанных в табл.~\ref{update-kubernetes} и \ref{update-opennebula}. \\
    \hline
    Возможность удаления параметров подключения. & Сохранённые ранее параметры удаляются с помощью запроса к REST API, описанного в табл.~\ref{delete-app}. \\
    \hline
    Возможность запроса текущей информации о приложении. & Текущая информация о приложении предоставляется по REST API с помощью запроса, описанного в табл.~\ref{get-app}, ответ на который описан в табл.~\ref{get-app-response}. \\
    \hline
    Возможность осуществления масштабирования. & Масштабирование приложения осуществляется по REST API с помощью запроса, описанного в табл.\ref{scale-app}. \\
    \hline
    Предоставление унифицированного интерфейса. & Сервис предоставляет унифицированный интерфейс, так как запросы текущей информации и масштабирования не зависят от управляемой платформы. \\
\end{xltabular}

Таким образом, все требования, предъявленные к сервису в пункте~\ref{requirements}, были реализованы в полном объёме.
